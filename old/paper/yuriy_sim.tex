To test the performance for inference of neural connectivity in a neural population, we simulated such inference in conditions close to such expected in real data.
Specifically, we solve connectivity inference for a network of simulated spiking neurons, constructed closely following empirical data known about the real neural networks in the cortex. We assume a population of neurons that are spontaneously firing action potentials.
While it is known, that the functional connectivity weights in general do not properly reflect anatomical connectivity in a circuit, we will show that for the above system of spontaneously firing neurons the direct correspondence, indeed, exists, and anatomical connectivity is properly recovered via functional connectivity weights.

Functional connectivity may fail to faithfully represent anatomical circuit structure if false correlations are present between different neurons, induced e.g. by common inputs, or if the dynamics of neural population is entirely concentrated on a low-dimensional subspace of the full configurational space $H$. Note that these two statements are, in a sense, different ways of stating the same condition: if activity of different neurons is tightly correlated, their dynamics is concentrated on a low-dimensional plane; and vice-versa - concentration of dynamics onto a low-dimensional plane will be perceived as correlation in activity of different neurons. (In turn, low dimensionality of the neural dynamics may be caused by different factors, including common input, small subset of command neurons driving the circuit, or even emergent property of a network.) Low dimensionality of the neural dynamics results in that the inference problem Eq.(\ref{eqn:loglik:definition}) becomes underdetermined, i.e. there may exist directions in $W$ along which connectivity is not constrained by activity data (i.e. directions orthogonal to the subspace of all observed neural activity configurations), or is poorly constrained. This, naturally, leads to $W$ being poorly defined along these directions. The necessary condition for good correspondence between functional connectivity weights $W$ and anatomical connectivity, therefore, is {\em full-dimensionality} of the observed neural dynamics. In case of spontaneously firing system of neurons this condition is, in fact, satisfied by many independent neuron-ignitions, thus, fully sampling possible directions in the configurational space $H$. If spontaneously active preparation by itself fails to display sufficient degree of independence between randomly firing neurons (e.g. if low-dimensionality of the activity subspace is the emergent property of studied circuit), such pattern may be induced by randomly activating subsets of neurons via ChR2 [...] or glutamate uncaging.

We also note that the correlations induced by secondary and so on synaptic transmissions (such as when neuron $A$ results in firing of neuron $B$, which in turn results in firing by neuron $C$), are all properly resolved in GLM-fitting process via the so called explaining-away process. In other words, because we do not just identify correlations between neural firings with the functional connectivity weights $w^{kk'}$, but instead statistically fit a model of neural interactions, if found weights between neurons $A$ and $B$, and $B$ and $C$ are sufficient to explain the correlation between $A$ and $C$, the weight connecting $A$ and $C$ will not appear in the model - the correlation between $A$ and $C$ was ``explained away'' by correlations between $A$ and $B$, and $B$ and $C$. By this, the multi-synaptic firing patterns do not confuse our estimation process.

We prepared a small network of $N=50$  neurons randomly sparsely connected, and firing stochastically with the base firing rate of about 5Hz. Each neuron was modeled with GLM as a linear-nonlinear driven Poisson spike generator, as described above
\begin{equation}\label{eqn:glm-def}
\begin{array}{l}
P(n^k_{t+1}|\{h_{t+1}\})=$Poiss$(\lambda^k_{t+1}\Delta t) \\
\lambda^k_{t+1}=g(J^k_t)=g(b_0+\sum_{\tau>0}w^{k,k'}_{\tau}n^{k'}_{t-\tau}),
\end{array}
\end{equation}
with exponential transfer function $g(J)=\exp(J)$. We assumed no external modulating input $X_{ext}(t)$.

The network was divided into excitatory and inhibitory components.
Neurons in such components were either entirely excitatory or inhibitory; i.e., all connections outgoing from such neuron were either all simultaneously positive (excitatory) or negative (inhibitory). Excitatory neurons were randomly connected with each other and the inhibitory neurons with probability $f_c=0.1$ (observed probability for a local connection between nearby neurons in the cortex [**]). The synaptic weight of each connection $v$, as defined by max EPSP amplitude, were generated from exponential distribution with mean $0.5 \mu V$ [**] (we neglected here the ``heavy tail'' of the distribution of synaptic weights observed in some datasets [**]).
While synaptic weights are typically measured in PSP units of $\mu V$,
in GLM model connections are measured in log-rate units of Eq.(\ref{eqn:glm-def}).
In other words, GLM weights describe the {\em change in probability of neuron $k$
to fire given neuron $k'$ has firing before}.
By utilizing this definition, we may convert a PSP weight $v_{EPSP}$ for a neuron $V_{base}$ below spike-triggering threshold into the GLM weight as $v_{GLM}\approx v_{EPSP}/V_{base}$. In other words, $V_{base}/v_{EPSP}$ spikes are required to push neuron over the threshold. Given the definition of the firing rate in (\ref{eqn:glm-def}), this
leads to the following equation for the log-rate couplings $w^{kk'}$
\begin{equation}\label{eqn:convert}
w^{kk'}=\ln(-\ln(\exp(-f^k \Delta t)-v^{kk'}_{EPSP}/V_{base})/\Delta t/f^k),
\end{equation}
where $f^k$ is the base ``stochastic'' firing rate of neuron $k$.

20\% of all neurons were taken to be inhibitory [**]. Interneurons were randomly connected among themselves and to excitatory neurons with the same frequency $f_c=0.1$ as above. The strength of the inhibitory connection was drawn from the exponential distribution with mean chosen to balance excitatory and inhibitory currents, and to achieve the final firing rate close to the base firing rate $f=g(b_0)$. Connection strengths were thus converted to log-rate units using Eq.(\ref{eqn:convert}), where $w^{kk'}$ was finally multiplied by -1 to reflect inhibition.

To generate a sequence of spikes in this population, we simulated activity of the network forward in time computing currents injected into each cell from all previous spikes of all neurons. Each spike was assumed to inject the same PSP waveform, described by a temporal filter $h_{PSP}$ [DIAGRAM] modeled as the difference of two exponentials with the rise time of $1ms$ and decay time of $10ms$ [**]
\begin{equation}
J^k_{t, inject} = \sum_{k'\neq k}w^{kk'}\sum_{\tau>0}
h_{PSP}(\tau) n^{k'}_{t-\tau}.
\end{equation}
(Given $1ms$ time step of our simulation, and the fact that
a signal in a local cortical circuit of the size of $\sim 1mm$
would suffer $\leq 1ms$ time lag, we neglect delays
in this simulation.) Additionally, each neuron exhibited refractory current with waveform $h_{REFR}$ [DIAGRAM] modeled with exponential with decay time of $5ms$ [**]
\begin{equation}
J^k_{t,{refr}} = \Omega^k \sum_{\tau>0}h_{REFR}(\tau) n^{k}_{t-\tau}.
\end{equation}
The network was then thus simulated forward in time with step of $\Delta t=1ms$. All neurons were assumed to be electro-physiologically similar (i.e. have the same PSP profiles, $\Omega$ and base firing rate, etc.) Spikes were generated at each time according to Bernoulli distribution
$n^k_t=$Bernoulli$[g(J^k_{t,spont}+J^k_{t,refr}+J^k_{t,inject})\Delta t]$.

Given a sequence of spikes, the fluorescence observations were generated using the setup in [VogPan]. Parameters for the model were chosen in accordance to our experience of analyzing a few actual cells [**]. Specifically, calcium decay time constant was $\approx 0.25 s$, the ratio of per-spike calcium influx to stochastic fluctuations in calcium concentration was $\approx3:1$,
and the background of calcium concentration was chosen to be about 30\% of per-spike calcium influx, thus corresponding to typical relatively low SNR of 1:3. Photon count per measurement was taken to be $\approx 3\cdot 10^4$. The population of cells was generated with these parameters while assuming that all parameters may vary by at least 30\% from cell to cell. Thus, we generated fluorescence for each cell using its unique cell of parameters, and produced sampled observations at frame-rate of 33Hz or 66Hz. [PARAMS-TABLE]
In this paper we develop a partial Bayesian approach for inferring functional connectivity in a population of spiking neurons observed using calcium imaging. While inferring functional connectivity from a set of simultaneous spike train recordings had been previously addressed for micro-electrode techniques \cite{PAN03c, TRUC05}, these approaches assumed observing the ``true'' spike trains of each observable neuron.  However, evidence suggests that even state-of-the-art spike sorting algorithms only obtain approximately $90\%$ accuracy in obtaining spike times \cite{HarrisBuzsaki00, WoodBlack08}.  With that in mind, \cite{Rigat06} developed an approach closely related to that here for such a scenario.  In particular, they also consider a sparse prior, use a Bernoulli GLM model, and develop a Metropolis-within-Gibbs sampler to approximate the necessary sufficient statistics for their model.  To our knowledge, we are the first address the problem of inferring connectivity from calcium population imaging data in a model based approach (though see \cite{Roxin08} for a related issue).  Because calcium imaging, in principle, has capacity to image populations of cells containing $\approx 10^3$--$10^6$ neurons, this opens the way for analysis of micro-circuitry in large and complete populations of neurons in neocortex or other brain areas.

The main challenge in this problem is indirect nature of calcium imaging data, which provides only noisy, low-pass nonlinear filtered, temporally sub-sampled observations of spikes of individual neurons. In order to find connectivity parameters, in a fully Bayesian settings, the hidden spike trains need to be integrated out. Obtaining a joint sample of unobserved spike trains, needed to compute relevant integrals, is a very non-trivial problem given high dimensionality of the hidden state (which scales at least linearly with $N$). In particular, methods for analyzing calcium imaging data for single neurons, \cite{Vogelstein2009}, do not generalize well for this application. To solve this problem, in this paper we develop a new method for obtaining spike train samples from a population of coupled low-dimensional HMMs by embedding sampling chains of states from individual low-dimensional HMM within a Gibbs sampler that loops in a predefined order over different coupled HMMs. The functional connectivity matrix is then inferred by maximizing the expected value of posterior log-likelihood in EM framework. An exponential prior is used to enforce the sparseness condition of the objective connectivity matrix, which significantly reduces the minimal amount of data required for a particular reconstruction accuracy.

By applying this method to observations of spontaneous activity in a simulated population of neurons, we can efficiently infer the functional connectivity matrix from only $\approx 10$ min of simulated calcium imaging data (c.f Figure \ref{fig:recvar-SNR}, \ref{fig:recvar-NT}). While the embedded-chain-within-Gibbs methods leads to an exact locally MAP estimate, under reasonable calcium imaging conditions we find that significant simplification is possible, where the posterior distribution may be assumed to approximately factorize, $P(\bX|\bF;\theta)\approx \prod P(X_i|F_i;\tilde\theta_i)$. This allows one to obtain samples of joint spike trains much more easily, and in parallel. Since the maximization procedure in EM also can be straightforwardly parallelized, thanks to special structure of the posterior likelihood, the entire inference method can be easily implemented as a highly parallelized application, offering significant savings in data-processing time. If calculations are performed on a large high-performance cluster, reconstruction of the connectivity matrix from $\approx 10$ min of calcium imaging data can be performed nearly in real time, by solving each neuron on a separate node and utilizing only about $10$ minutes of computational time per each node. This is an important virtue of our method.

The above described approach differs significantly from naively computing pairwise measures, such as correlation coefficients, which are far faster to compute.  In particular, our inferred weights depend on \emph{all} the observed neurons, not simply the pairwise measures.  In other words, the sufficient statistic for $\bw_i$ is $P[X_i(t), \bX(t-\Delta) | \bF]$, as compared with correlation-like measures that depend only on $P[X_i(t) | X_j(t-\Delta)]$.  So, while a naive application of correlation coefficients might be attractive here, the resulting estimated ``weights'' are effectively useless (not shown).  

%In the view of these achievements and failures, 
A number of possible improvements of our method can be proposed. One of the biggest challenges for inferring neural connectivity from functional data is the presence of so called hidden inputs from unobserved neurons \cite{Vidne08}. Since it is typically impossible to expect that activity of all neurons in a closed neural circuit can be monitored, such hidden inputs should be always anticipated in real imaging data. Correlations in hidden inputs are capable of successfully mimicking functional connections among different observed neurons, thus presenting a substantial challenge for estimating neural circuit's connectivity from activity observations alone. Developing methods to cope with such hidden inputs is currently area of active research \cite{Vidne08}.  %Incorporating these features into our model is an important direction for future work. 

Along with investigation of ways to combat the effect of unobserved neurons, we have considered several other potential directions for future improvements of our method.  Incorporating photo-stimulation to activate or deactivate individual neurons or sub-populations may be used to significantly increase statistical power of a given set of observations \cite{Deisseroth05,SzobotaIsacoff07}. %Especially, such external stimulation of network activity may be helpful in the case where the natural activity of a circuit results in strongly correlated firing patterns [Rafa?]. 
Although naturally observed activity may not allow reliable determination of circuit's connectivity matrix, by utilizing external stimulation, a sufficiently rich sample of activity patterns may be collected, and true anatomical structure of the neural circuit may be inferred. Developing the optimal sequences of artificial stimuli and their implementation in the actual experiments are other important directions for future work.

Furthermore, improvements of the algorithms for faster implementation of both the E and M step of our algorithm are under development.  Specifically, fast, non-negative deconvolution of calcium to infer spike trains is a promising alternative \cite{Vogelstein08}.  This fast algorithm utilizes the tridiagonal structure of the state-space problem, a interior-point approach to impose the non-negativity constraint, and Gaussian elimination to ascend the concave likelihood surface in $O(T)$ \cite{Pan08b}.  This approach only yields the MAP estimate of the most likely spike train, and so a true maximization step (integrating over a distribution) is not feasible.  However, approximating the posterior with the MAP seems sufficient when SNR is sufficiently high \cite{Vogelstein08}.

Several improvements in the model are also under investigation.  Explicitly modeling Poisson statistics of the spike counts in time-bins with large width, $\Delta$, rather than Bernoulli distribution used in this work, might helpful for fast spiking neurons. Modifications of our generative model allowing to deal with fluorescent signal non-stationarities, e.g. due to dye bleaching and drift, will be important to reliably apply our method to real data. 

Finally, a fully Bayesian algorithm for estimating the posterior distributions of all the parameters, as opposed to only finding the MAP estimate, is of great interest. Such fully-Bayesian extension is conceptually simple: we just need to extend this work's Gibbs sampler to additionally sample from $\bth$ given the spike trains $\bX$. Since we already have a method for drawing the spike trains $\bX$ given $\bth$ and $\bF$, with such an additional sampler we may obtain samples from $P(\bX,\bth | \bF)$ simply by sampling from $\bX \sim P(\bX|\theta,\bF)$ and $\bth \sim P(\bth |\bX)$ within a Gibbs sampling procedure.  Sampling from the posteriors $P(\bth|\bX)$ in the GLM setting is tractable using hybrid Monte Carlo methods, since all the posteriors are log-concave \cite{Ishwaran99,Gamerman97,Gamerman98,Yashar08}.  All these advances are currently being pursued.
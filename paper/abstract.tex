Simultaneously imaging small populations of neurons using calcium sensors is now becoming routine in labs across the world.  We have developed analytical tools to maximally utilize these beautiful data sets.  In particular, the following neurobiological questions are of interest: (i) what are the response properties of populations of neurons that are in close physical proximity, (ii) what is the functional circuit underlying these response properties.  Previously, we built a forward model characterizing the relationship between stimuli, spike trains, intracellular calcium concentration, and fluorescence time series observations; and then inverted that model using non-linear state-space methods (i.e., a particle-filter-smoother adapted for this model, embedded in an expectation-maximization algorithm).  

Here, that forward model is generalized to allow for dependencies between neurons (i.e., cross-coupling terms), and the inference and learning algorithm is appropriately modified as well.  We show that given only a short movie (< 10 min), and reasonable assumptions on noisiness, spike rates, and coupling strengths, we can accurately reconstruct circuits governing small populations (e.g., 10 neurons; see S1).  As the number of neurons increases, or the data quality decreases, we can impose experimentally justifiable priors on the distribution of coupling weights, to improve our estimates.  

We are currently pursuing confirming our parameter estimates using in vitro preparations.  In particular, by simultaneously imaging a population of neurons, and recording electrophysiologically from at least one, we can validate our inferred connection strengths.  Our hope is that they algorithms developed here will be useful in a wide array of experiments and preparations, especially in vivo calcium imaging.
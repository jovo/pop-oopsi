Since Ramon y Cajal discovered that the brain is not a syncytium, but rather a rich and dense \emph{network} of neurons, neuroscientists have wondered about the details of these networks.  Since then, while much has been learned about ``macro-circuits''  --- the connectivity between populations of neurons --- relatively little is known about ``micro-circuits'' --- the connectivity within populations of neurons. Broadly, one can imagine two distinct strategies for inferring microcircuit connectivity: anatomical and functional.  While anatomical approaches are rapidly improving and currently show great promise \cite{Briggman2006, Brainbow07, LuLichtman09}, our focus here is on functional approaches.

Experimental tools that enable approximately simultaneous observations of the activity many (e.g., $O(10^2)$) neurons are now widely available.  While arrays of extracellular electrodes have been exploited for this purpose, they are inadequate for inferring monosynaptic connectivity, as the inter-electrode spacing is typically too large.  Alternately, calcium-sensitive fluorescent indicators provide a glimpse into the spiking activity of many neighboring neurons \cite{Tsien89}, which are more likely to be connected \cite{Abeles91, Braitenberg1998}. Some organic dyes achieve signal-to-noise ratios (SNRs) yielding single spike resolution \cite{ImagingManual}.  In combination with these dyes, bulk-loading techniques enable experimentalists to simultaneously fill populations of neurons with such dyes \cite{StosiekKonnerth03}.  While these approaches are state-of-the-art in terms of SNR, their invasiveness is a significant drawback.  To that end, genetically encoded calcium indicators are under rapid development from a number of groups, and they are approaching SNR levels of nearly single spike accuracy as well \cite{WallaceHasan08}. Regardless of the source of the fluorescence, microscopy technologies for collecting the signal are also rapidly developing.  Cooled CCDs for wide-field imaging (either epifluorscence or confocal) now achieve a quantum efficiency of $\approx 90 \%$ with frame rates easily exceeding $30$ or $60$ Hz \cite{Djurisic04}.  For in vivo work, 2-photon laser scanning microscopy can achieve similar frame rates, by designing software to efficiently control the typical scanners (Valentin Naegerl, Tom Mrsic-Flogel, and Bruno Pichler, personal communications), using acoustic-optical deflectors to focus light at arbitrary locations in (three-dimensional) space \cite{ReddySaggau05, Iyer06, SalomeBourdieu06, ReddySaggau08}, or using resonant scanners \cite{NguyenParker01}.  Together, these experimental tools can provide movies indicating calcium based fluorescent transients for small populations of neurons (e.g., $O(10^2)$), with ``reasonable'' SNR, at 30 Hz, both in the in vitro and in vivo scenarios.  

Given these experimental advances in functional neural imaging, the stage is set for the development of complementary computational tools.  We first define a coupled hidden Markov Model, relating the observed variables (fluorescence traces from observable neurons) to the hidden variables (spike trains of those neurons), as governed by a small set of parameters, including the functional connectivity matrix.  The sufficient statistics for this matrix is joint posterior distribution of all the hidden states.  While in previous work \cite{VogPan09} we developed a sequential Monte Carlo (SMC) algorithm to optimally infer spike trains given the fluorescence trace of a single neuron, it is well known in the statistics community that this approach does not scale well as the dimensionality of the hidden state increases \cite{DFG01}.  Thus, we extend this method by enveloping our SMC strategy within a Gibbs sampler.  In other words, we recursively generate samples of spike trains for each neuron, conditioned on all other neurons.  Note that the Gibbs approach is appropriate here, since it is well-known that Gibbs sampling is most efficient in the limit of weakly-coupled random variables \cite{Gelman03, RC05}.  However, our SMC approach alone does not provide unbiased samples of spike trains, due to its discretization bias \cite{DFG01}.  We can eliminate this discretization bias by embedding the discrete SMC results within a continuous state space \cite{Neal03}.  Neal \cite{NBR03} shows that this approach indeed form a Markov chain with the desired equilibrium density.  Given these unbiased samples of joint spike trains, we have the sufficient statistics necessary to estimate the functional connectivity matrix.  We then show that by making an approximate factorization (akin to variational inference \cite{WainwrightJordan08}), we can significantly expedite the computations; 10 minutes of data imaged at 30 Hz from 200 neurons  requires only about 10 minutes of processing, while running on an appropriate cluster.  In simulations, given reasonable assumptions about parameters, noise, imaging rate, etc, we can estimate the functional connectivity matrix quite well. We then point out the limitations of this approach, with respect to observation noise, experimental duration, imaging rate, number of neurons, and certain  model misspecifications.  By imposing biophysically based priors on some of our model parameters, we can further improve this technique.  Thus, this approach appears to be a promising technique for the inference of functional connectivity matrices for small populations of neurons.  